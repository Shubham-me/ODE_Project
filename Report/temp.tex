\documentclass{article}
\usepackage{amsmath}
\usepackage{tikz}

\begin{document}

\section{Solving for Square wave function}

The objective is to solve the differential equation
\[ LC \frac{d^2 v_0}{dt^2} + \frac{L}{R} \frac{dv_0}{dt} + v_0 = v_p(t) \]
where \( v_p(t) \) is a square wave function defined as follows:

\[
v_p(t) =
\begin{cases}
V_{\text{DC}} & \text{if } 0 \leq t < D \cdot T_{\text{SW}} \\
0 & \text{if } D \cdot T_{\text{SW}} \leq t < T_{\text{SW}}
\end{cases}
\]

This function is periodic with period \( T_{\text{SW}} \).

\begin{figure}[h]
\centering
\begin{tikzpicture}
% Axis
\draw[->] (-0.5,0) -- (6,0) node[right] {\( t \)};
\draw[->] (0,-0.5) -- (0,2) node[above] {\( v_p(t) \)};
% Function
\draw[thick] (0,1.5) -- (2,1.5) -- (2,0) -- (4,0) -- (4,1.5) -- (6,1.5);
% Period markings
\draw[dashed] (2,0) -- (2,-0.2) node[below] {\( D \cdot T_{\text{SW}} \)};
\draw[dashed] (4,0) -- (4,-0.2) node[below] {\( T_{\text{SW}} \)};
\end{tikzpicture}
\caption{Square wave function \( v_p(t) \) with period \( T_{\text{SW}} \) and amplitude \( V_{\text{DC}} \)}
\label{fig:square_wave}
\end{figure}

We have the following initial conditions for the differential equation:
\[ v_0(0) = 0 \]
\[ \frac{dv_0}{dt}(0) = 0 \]

We convert the square wave to its Fourier series, which is equal to:

\[
D \cdot V_{\text{DC}} + \sum_{n=1}^{\infty} \left( A_n \cos\left(\frac{2\pi n t}{T_{\text{SW}}}\right) + B_n \sin\left(\frac{2\pi n t}{T_{\text{SW}}}\right) \right)
\]

where \( A_n \) and \( B_n \) are the Fourier coefficients, and \( V_{\text{DC}} \) is the amplitude of the function.
\(A_n\) and \(B_n\) are given by the following equations:

\[A_n = \frac{2}{T_{\text{SW}}} \int_{0}^{T_{\text{SW}}} v_p(t) \cos\left(\frac{2\pi n t}{T_{\text{SW}}}\right) \, dt = V_{\text{DC}} \frac{\sin\left(2 \pi n D\right)}{\pi n}\]
\[B_n = \frac{2}{T_{\text{SW}}} \int_{0}^{T_{\text{SW}}} v_p(t) \sin\left(\frac{2\pi n t}{T_{\text{SW}}}\right) \, dt = V_{\text{DC}} \frac{1 - \cos\left(2 \pi n D\right)}{\pi n}= V_{\text{DC}} \frac{2 \sin^2\left(\pi n D\right)}{\pi n}\]

We approximate this by the truncated Fourier series:

\[
v_{p,N}(t) = D \cdot V_{\text{DC}} + \sum_{n=1}^{N} \left( A_n \cos\left(\frac{2\pi n t}{T_{\text{SW}}}\right) + B_n \sin\left(\frac{2\pi n t}{T_{\text{SW}}}\right) \right)
\]

where \( N \) is the number of terms used in the approximation.

The equation to solve is:

\[
LC \frac{d^2 v_0}{dt^2} + \frac{L}{R} \frac{dv_0}{dt} + v_0 = v_{p,N}(t)
\]

with initial conditions:

\[
v_{0,N}(0) = 0, \quad \frac{dv_{0,N}}{dt}(0) = 0
\]

where \( v_{p,N}(t) \) is the truncated Fourier series of \( v_p(t) \) with \( N \) terms,
and \( v_{0,N}(t) \) is the solution to the differential equation with \( v_{p,N}(t) \).

We have the equation:
\begin{equation}
v_{0,N}(t) = v_{0,N,\text{particular}}(t) + v_{0,N,\text{homogeneous}}(t)
\end{equation}

Now, let's consider the homogeneous solution \(v_{0,N,\text{homogeneous}}(t)\). 

The homogeneous equation is given by:
\[
LC \frac{d^2 v_0}{dt^2} + \frac{L}{R} \frac{dv_0}{dt} + v_0 = 0
\]

Consider the solution space of this equation. Let \(v_{01}(t)\) and \(v_{02}(t)\) be solutions. Then \(v_{01}(t) - v_{02}(t)\) is a solution of the homogeneous equation.

To find solutions of the homogeneous equation, we consider the characteristic equation:
\[
LC k^2 + \frac{L}{R} k + 1 = 0
\]

Depending on the roots of this equation, the solutions of the homogeneous equation vary:
\begin{enumerate}
    \item If both roots are real and distinct (\( \alpha \) and \( \beta \)), then \( e^{\alpha t} \) and \( e^{\beta t} \) are solutions.
    \item If both roots are real and equal (\( r \)), then \( e^{rt} \) and \( t e^{rt} \) are solutions.
    \item If the roots are complex (\( \alpha \pm i \beta \)), then \( e^{\alpha t} \sin(\beta t) \) and \( e^{\alpha t} \cos(\beta t) \) are solutions.
\end{enumerate}

These solutions form a vector space with dimension 2, and the basis vectors (solutions) depend on the values of \(LC\), \(L/R\), and \(1\).

Now, we have to solve for \(v_{0,N,\text{particular}}(t)\), also called the particular solution of \(v_{0,N}\).

In the context of our problem, the particular solution \( v_{0,N,\text{particular}} \) is expressed as the sum of \( \widetilde{v}_0 \) and a series of terms \( \widetilde{v}_n \) for \( n = 1, 2, \ldots, N \).
\[ v_{0,N,\text{particular}} = \widetilde{v}_0 + \sum_{n=1}^{N} \widetilde{v}_n \]

where \( {v}_{0,N,\text{particular}}(t) \) is the particular solution of \(LC \frac{d^2 v_0}{dt^2} + \frac{L}{R} \frac{dv_0}{dt} + v_0 = v_{p,N}(t)\), \( \widetilde{v}_0 \) is the particular solution of \(LC \frac{d^2 v_0}{dt^2} + \frac{L}{R} \frac{dv_0}{dt} + v_0 = D \cdot V_{\text{DC}}\), and \( \widetilde{v}_n \) is the particular solution of \(LC \frac{d^2 v_0}{dt^2} + \frac{L}{R} \frac{dv_0}{dt} + v_0 = A_n \cos\left(\frac{2 \pi n t}{T_{\text{SW}}}\right) + B_n \sin\left(\frac{2 \pi n t}{T_{\text{SW}}}\right)\).

Calculating the particular solution of \(LC \frac{d^2 v_0}{dt^2} + \frac{L}{R} \frac{dv_0}{dt} + v_0 = D \cdot V_{\text{DC}}\), we have the following cases:
\begin{itemize}
    \item \(LC \neq 0\)
    \[
        \widetilde{v}
        _{0}(t) = D \cdot V_{\text{DC}}
    \]
\end{itemize}
Now, Calculating the particular solution of \(LC \frac{d^2 v_0}{dt^2} + \frac{L}{R} \frac{dv_0}{dt} + v_0 = A_n \cos\left(\frac{2 \pi n t}{T_{\text{SW}}}\right) + B_n \sin\left(\frac{2 \pi n t}{T_{\text{SW}}}\right)\)\\
$\widetilde{v}_n = \alpha_n\sin\left(\frac{2\pi nt}{T_{\text{SW}}}\right) + \beta_n \cos\left(\frac{2\pi nt}{T_{\text{SW}}}\right)$
where
\[
\alpha_n = \frac
{
\begin{vmatrix}
B_n & -\frac{L}{R}\cdot 2\pi n \\
A_n & LC-\left(\frac{2\pi n}{T_{\text{SW}}}\right)^2
\end{vmatrix}
}
{
\begin{vmatrix}
LC-\left(\frac{2\pi n}{T_{\text{SW}}}\right)^2 & -\frac{L}{R}\cdot 2\pi n \\
\frac{L}{R}\cdot 2\pi n & LC-\left(\frac{2\pi n}{T_{\text{SW}}}\right)^2
\end{vmatrix}   
} \quad,
\beta_n = \frac
{
\begin{vmatrix}
LC-\left(\frac{2\pi n}{T_{\text{SW}}}\right)^2 & B_n \\
-\frac{L}{R}\cdot 2\pi n & A_n
\end{vmatrix}
}
{  
\begin{vmatrix}
LC-\left(\frac{2\pi n}{T_{\text{SW}}}\right)^2 & -\frac{L}{R}\cdot 2\pi n \\
\frac{L}{R}\cdot 2\pi n & LC-\left(\frac{2\pi n}{T_{\text{SW}}}\right)^2
\end{vmatrix}
}.
\]

\end{document}
